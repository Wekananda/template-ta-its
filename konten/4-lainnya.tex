\section{Jadwal Kegiatan}
Berikut ini disajikan tabel jadwal kegiatan yang akan dilakukan selama 3 bulan yang berkoresponden dengan metode penelitian.
% Ubah tabel berikut sesuai dengan isi dari rencana kerja
\newcommand{\w}{}
\newcommand{\G}{\cellcolor{gray}}
\begin{table}[h!]
  \begin{tabular}{|p{3.5cm}|c|c|c|c|c|c|c|c|c|c|c|c|c|c|c|c|}

    \hline
    \multirow{2}{*}{Nama Kegiatan} & \multicolumn{16}{|c|}{Minggu Ke-} \\
    \cline{2-17} &
    1 & 2 & 3 & 4 & 5 & 6 & 7 & 8 & 9 & 10 & 11 & 12 & 13 & 14 & 15 & 16 \\
    \hline

    % Gunakan \G untuk mengisi sel dan \w untuk mengosongkan sel
    Pengumpulan Data &
    \G & \G & \w & \w & \w & \w & \w & \w & \w & \w & \w & \w & \w & \w & \w & \w \\
    \hline

    Analisis dan Perancangan Sistem &
    \w & \w & \G & \G & \G & \G & \G & \G & \G & \w & \w & \w & \w & \w & \w & \w \\
    \hline
    
    Simulasi Sistem &
    \w & \w & \w & \w & \w & \w & \w & \w & \w & \G & \w & \w & \w & \w & \w & \w \\
    \hline
    
    Desain Uji Coba Sistem &
    \w & \w & \w & \w & \w & \w & \w & \w & \w & \w & \G & \w & \w & \w & \w & \w \\
    \hline

    Analisis dan Evaluasi Hasil&
    \w & \w & \w & \w & \w & \w & \w & \w & \w & \w & \w & \G & \G & \G & \w & \w \\
    \hline
    
    Penarikan Kesimpulan dan Saran &
    \w & \w & \w & \w & \w & \w & \w & \w & \w & \w & \w & \w & \w & \w & \G & \G \\
    \hline

  \end{tabular}
\end{table}

\newpage
