\addcontentsline{toc}{chapter}{LEMBAR PENGESAHAN}
\begin{center}
	\large
  \textbf{LEMBAR PENGESAHAN}
\end{center}
\vspace{2ex}
% Menyembunyikan nomor halaman
\thispagestyle{empty}

\begin{center}
  % Ubah kalimat berikut dengan judul tugas akhir
  \MakeUppercase{\textbf{Peningkatan Akurasi Perhitungan Jumlah Kendaraan menggunakan model StrongSORT}}
\end{center}
\vspace{1ex}
\begingroup
  % Pemilihan font ukuran small
  \small
\begin{spacing}{1.5}
  \begin{center}
    % Ubah kalimat berikut dengan pernyataan untuk lembar pengesahan
   \MakeUppercase{\textbf{Proposal Tugas Akhir}}
   
   
   Diajukan untuk memenuhi salah satu syarat
   
   
   memperoleh gelar Sarjana pada
   
   
   Program Studi S-1 Matematika
   
   
   Departemen Matematika
   
   
   Fakultas Sains dan Analitika Data
   
   
   Institut Teknologi Sepuluh Nopember
  \end{center}
 
\vspace{3ex}
  \begin{center}
    % Ubah kalimat berikut dengan nama dan NRP mahasiswa
    Oleh: \textbf{I Gusti Agung Vivekananda} \\
    NRP. 06111940000111
  \end{center}

\vspace{2ex}
  \begin{center}
    Disetujui oleh Tim Penguji Proposal Tugas Akhir :
  \end{center}
  \vspace{2ex}
  \begingroup
    % Menghilangkan padding


    \noindent
   
     \begin{enumerate}
      % Ubah kalimat-kalimat berikut dengan nama dan NIP dosen pembimbing pertama
     \item Dr. Budi Setiyono, S.Si, MT  \hspace{60mm}  Pembimbing
     \item  Dr. Dwi Ratna Sulistyaningrum, S.Si, MT  \hspace{39mm} Penguji
     \item Mohammad Iqbal, S.Si., M.Si, Ph.D \hspace{48mm} Penguji
  \end{enumerate}
    
 
  \endgroup

\begin{center}
  \vspace{40ex}

 \MakeUppercase{\textbf{Surabaya}}
 
 
 \textbf{Agustus, 2022}

\end{center}
 \end{spacing}
\newpage