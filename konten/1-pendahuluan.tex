\chapter{PENDAHULUAN}
\section{Latar Belakang}

% Ubah paragraf-paragraf berikut sesuai dengan latar belakang dari tugas akhir
\lipsum[2-4]

\section{Rumusan Masalah}

% Ubah paragraf berikut sesuai dengan rumusan masalah dari tugas akhir
Rumusan masalah yang diangkat dalam penelitian ini adalah sebagai berikut: 
\begin{enumerate}
  \item Bagaimana implementasi model StrongSORT untuk meningkatkan akurasi dalam menghitung jumlah kendaraan yang melintas?
  \item Bagaimana meningkatkan akurasi model StrongSORT dalam menghitung jumlah kendaraan yang melintas?
\end{enumerate}

\section{Batasan Masalah}

% Ubah paragraf berikut sesuai dengan penelitian lain yang terkait dengan tugas akhir
Batasan masalah yang digunakan dalam penelitian ini adalah sebagai berikut: \begin{enumerate}
  \item Pengambilan data dilakukan pada pagi hari, siang hari dan sore hari
  \item Data pengujian yang digunakan sebagai bahan pengolahan citra berupa rekaman video CCTV pada kota Surabaya
  \item Kendaraan yang dapat dihitung terbatas pada 5 kategori kendaraan, yaitu Sepeda, Motor, Mobil, Bus dan truk
  \item Peningkatan akurasi akan dibandingkan dengan sistem penghitungan kendaraan dengan YOLOv7 tanpa menggunakan StrongSORT sebagai MOT
\end{enumerate}

\section{Tujuan Penelitian}

% Ubah paragraf berikut sesuai dengan tujuan penelitian dari tugas akhir
Tujuan penelitian yang akan dicapai dalam penelitian ini adalah sebagai berikut:
\begin{enumerate}
    \item Mengimplementasikan model StrongSORT untuk meningkatkan akurasi penghitungan jumlah kendaraan terhadap kendaraan yang melintas.
    \item Mengetahui hasil berupa peningkatan akurasi dari penerapan model StrongSORT dalam menghitung jumlah kendaraan yang melintas.
\end{enumerate}

\section{Manfaat Penelitian}
Terdapat dua manfaat yang dihasilkan dalam penelitian ini yaitu secara teori dan pratik. Secara teori, penelitian ini memberikan kontribusi penelitian pada bidang \emph{computer vision} yang dapat digunakan sebagai referensi pada penelitian berikutnya. Secara praktik, penelitian ini dapat berkontribusi pada \emph{Intelligent Transportation System} dalam mengumpulkan data berupa jumlah kendaraan yang melintas pada suatu jalan.
\newpage 