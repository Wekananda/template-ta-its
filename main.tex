% Pengaturan ukuran teks dan jenis dokumen
\documentclass[12pt]{report}

% Pengaturan ukuran halaman dan margin
\usepackage[a4paper,top=30mm,left=30mm,right=20mm,bottom=25mm]{geometry}

% Pengaturan ukuran spasi
\usepackage[singlespacing]{setspace}

% Font Times New Roman
\usepackage{newtxtext,newtxmath}

% Judul dokumen
\title{Proposal Tugas Akhir ITS Departemen Matematika}
\author{I Gusti Agung Vivekananda}

% Pengaturan format bahasa
\usepackage[indonesian]{babel}

% Pengaturan detail pada file PDF
\usepackage[pdfauthor={\@author},bookmarksnumbered,pdfborder={0 0 0}]{hyperref}

% Pengaturan jenis karakter
\usepackage[utf8]{inputenc}

\usepackage[T1]{fontenc}

% Pengaturan ukuran indentasi
\setlength{\parindent}{2em}

% Pengaturan kutipan artikel
%\usepackage[numbers]{natbib}

\usepackage{csquotes}
\usepackage[style=apa, backend=biber]{biblatex}
\DeclareLanguageMapping{indonesian}{american-apa}

\addbibresource{./pustaka/pustaka.bib}

% Package lainnya
\usepackage{etoolbox} % Mengubah fungsi default
\usepackage{enumitem} % Pembuatan list
\usepackage{lipsum} % Pembuatan template kalimat
\usepackage{graphicx} % Input gambar
\usepackage{longtable} % Pembuatan tabel
\usepackage[table,xcdraw]{xcolor} % Pewarnaan tabel
%\usepackage[numbers]{natbib} % Kutipan artikel
\usepackage{changepage} % Pembuatan teks kolom
\usepackage{multicol} % Pembuatan kolom ganda
\usepackage{multirow} % Pembuatan baris ganda
\usepackage{wrapfig}
\usepackage{tabularx}
%\usepackage{lmodern} plus symbol gone because this
\usepackage{eso-pic}
\usepackage{float}

% Pengaturan penomoran halaman
\usepackage{fancyhdr}
\fancyhf{}
\renewcommand{\headrulewidth}{0pt}
\pagestyle{fancy}
\fancyfoot[CE,CO]{\thepage}
\patchcmd{\chapter}{plain}{fancy}{}{}
\patchcmd{\chapter}{empty}{plain}{}{}

% Pengaturan format judul bab
\usepackage{titlesec}
\titleformat{\chapter}{\bfseries\Large}{\Roman{chapter}.}{1ex}{\vspace{0ex}}
\titleformat{\section}{\bfseries\large}{\MakeUppercase{\thesection}}{1ex}{\vspace{1ex}}
\titleformat{\subsection}{\bfseries\large}{\MakeUppercase{\thesubsection}}{1ex}{}
\titleformat{\subsubsection}{\bfseries\large}{\MakeUppercase{\thesubsubsection}}{1ex}{}
\titlespacing{\chapter}{0ex}{0ex}{4ex}
\titlespacing{\section}{0ex}{1ex}{0ex}
\titlespacing{\subsection}{0ex}{0.5ex}{0ex}
\titlespacing{\subsubsection}{0ex}{0.5ex}{0ex}

% Isi keseluruhan dokumen
\begin{document}

  % Menonaktifkan penomoran halaman
 \pagenumbering{gobble}

  % Lembar pengesahan
  %\input{pengesahan/pengesahan.tex}
  %\newpage
  
  %cover
   \begin{flushleft} 
    \includegraphics[width=25mm]{./gambar/logo-its.png}\\[10mm]{\definecolor{blue}{RGB}{0 103 172}{\color{blue}\makebox[\textwidth]{\rule{200cm}{10mm}}}}\\
    \vspace{10mm}\noindent\setstretch{1.5}\textsf{%
    {\fontsize{14}{2}\selectfont{\textbf{PROPOSAL TUGAS AKHIR~ - ~KM184801}}}\\
    \vspace{2cm}
    {\fontsize{18}{2}\selectfont{\textbf{\MakeUppercase{Peningkatan Akurasi Perhitungan Jumlah Kendaraan menggunakan model StrongSORT}}}}\\
    \vspace{1.5cm}
    {\fontsize{14}{2}\selectfont{\textbf{\MakeUppercase{I Gusti Agung Vivekananda}}}
    }\\{\fontsize{14}{2}\selectfont{NRP 06111940000111}}\\
    \vspace{20mm}
    {\fontsize{14}{2}\selectfont{Dosen Pembimbing}}\\
    {\fontsize{14}{2}\selectfont{\textbf{Dr. Budi Setiyono, S.Si, MT}}}\\
    {\fontsize{14}{2}\selectfont{NIP 19720207 199702 1 001}}\\
    \vspace{1.5cm}
    {\fontsize{12}{2}\textbf{Program Studi Sarjana}}\\
    {\fontsize{12}{2}Departemen~ {Matematika}}\\
    {\fontsize{12}{2}{Fakultas}~ {Sains dan Analitika Data}}\\
    {\fontsize{12}{2}{Institut Teknologi Sepuluh Nopember}}\\
    {\fontsize{12}{2}{Surabaya}}\\
    \fontsize{12}{2}{2022}}
    \vspace{2cm}
    \end{flushleft}
  
  \setcounter{page}{1}
  
    % Pengaturan ukuran indentasi paragraf
  \setlength{\parindent}{2em}

  % Pengaturan ukuran spasi paragraf
  %\setlength{\parskip}{1ex}

  \pagenumbering{roman}
  \addcontentsline{toc}{chapter}{LEMBAR PENGESAHAN}
\begin{center}
	\large
  \textbf{LEMBAR PENGESAHAN}
\end{center}
\vspace{2ex}
% Menyembunyikan nomor halaman
\thispagestyle{empty}

\begin{center}
  % Ubah kalimat berikut dengan judul tugas akhir
  \MakeUppercase{\textbf{Peningkatan Akurasi Perhitungan Jumlah Kendaraan menggunakan model StrongSORT}}
\end{center}
\vspace{1ex}
\begingroup
  % Pemilihan font ukuran small
  \small
\begin{spacing}{1.5}
  \begin{center}
    % Ubah kalimat berikut dengan pernyataan untuk lembar pengesahan
   \MakeUppercase{\textbf{Proposal Tugas Akhir}}
   
   
   Diajukan untuk memenuhi salah satu syarat
   
   
   memperoleh gelar Sarjana pada
   
   
   Program Studi S-1 Matematika
   
   
   Departemen Matematika
   
   
   Fakultas Sains dan Analitika Data
   
   
   Institut Teknologi Sepuluh Nopember
  \end{center}
 
\vspace{3ex}
  \begin{center}
    % Ubah kalimat berikut dengan nama dan NRP mahasiswa
    Oleh: \textbf{I Gusti Agung Vivekananda} \\
    NRP. 06111940000111
  \end{center}

\vspace{2ex}
  \begin{center}
    Disetujui oleh Tim Penguji Proposal Tugas Akhir :
  \end{center}
  \vspace{2ex}
  \begingroup
    % Menghilangkan padding


    \noindent
   
     \begin{enumerate}
      % Ubah kalimat-kalimat berikut dengan nama dan NIP dosen pembimbing pertama
     \item Dr. Budi Setiyono, S.Si, MT  \hspace{60mm}  Pembimbing
     \item  Dr. Dwi Ratna Sulistyaningrum, S.Si, MT  \hspace{39mm} Penguji
     \item Mohammad Iqbal, S.Si., M.Si, Ph.D \hspace{48mm} Penguji
  \end{enumerate}
    
 
  \endgroup

\begin{center}
  \vspace{40ex}

 \MakeUppercase{\textbf{Surabaya}}
 
 
 \textbf{Agustus, 2022}

\end{center}
 \end{spacing}
\newpage
  \begin{center}
  \large\textbf{Peningkatan Akurasi Perhitungan Jumlah Kendaraan menggunakan model StrongSORT}
\end{center}

\addcontentsline{toc}{chapter}{ABSTRAK}

\vspace{2ex}


\begingroup
  \setlength{\tabcolsep}{0pt}
  \noindent
  \textbf{
  \begin{tabularx}{\textwidth}{l >{\centering}m{2em} X}
    Nama Mahasiswa / NRP        &:& I Gusti Agung Vivekananda / 06111940000111 \\
    Departemen       &:&	Matematika FSAD - ITS \\
    Dosen Pembimbing  &:& Dr. Budi Setiyono, S.Si, MT \\
  \end{tabularx}
  }
\endgroup

\vspace{6ex}

\noindent\textbf{Abstrak}
\vspace{1ex}


% Ubah paragraf berikut dengan abstrak dari tugas akhir
Perkembangan jumlah kendaraan di Indonesia terus meningkat. Hal ini menyebabkan kemacetan dan penambahan jumlah emisi yang menjadi kontributor utama terhadap perubahan iklim global. Salah satu cara menyelesaikan permasalahan tersebut adalah dengan menerapkan \emph{Intelligent Transportation Systems} (ITS). Perhitungan jumlah kendaraan merupakan komponen penting dalam ITS karena jumlah kendaraan memberikan informasi penting mengenai arus lalu lintas seperti kondisi lalu lintas, okupansi lajur, dan tingkat kemacetan, yang dapat dimanfaatkan untuk peringatan kecelakaan, dan pencegahan kemacetan. Penelitian ini mengusulkan untuk mengembangkan penghitungan jumlah kendaraan dengan menggunakan model StrongSORT untuk melakukan pelacakan pada objek kendaraan dan YOLOv7 untuk mendeteksi objek kendaraan. Data latih yang digunakan untuk pendeteksi objek diperoleh dari \emph{Common Objects in Context} (COCO), OpenImage, dan CCTV di kota Surabaya. Hasil penelitian ini akan dibandingkan dengan model penghitungan kendaraan menggunakan YOLOv7.

\vspace{4ex}

% Ubah kata-kata berikut dengan kata kunci dari tugas akhir
\noindent\textbf{Kata Kunci: YOLOv7, StrongSORT, \emph{Computer Vision}, \emph{Multi Object Tracking}, \emph{Intelligent Transportation Systems}}

  \cleardoublepage
  
%  \input{konten/lembar-pengesahan-2}
  % Daftar isi
  \renewcommand*\contentsname{DAFTAR ISI}
  \addcontentsline{toc}{chapter}{\contentsname}
  \tableofcontents
  \cleardoublepage

  % Daftar gambar
  \renewcommand*\listfigurename{DAFTAR GAMBAR}
  \addcontentsline{toc}{chapter}{\listfigurename}
  \listoffigures
  \cleardoublepage

  %\renewcommand*\listtablename{DAFTAR TABEL}
  %\listoftables
  %\addcontentsline{toc}{chapter}{\listtablename}
  %\cleardoublepage
  
   % Nomor halaman isi dimulai dari sini
  \pagenumbering{arabic}
  

  % Konten pendahuluan
  \chapter{PENDAHULUAN}
\section{Latar Belakang}

% Ubah paragraf-paragraf berikut sesuai dengan latar belakang dari tugas akhir
\lipsum[2-4]

\section{Rumusan Masalah}

% Ubah paragraf berikut sesuai dengan rumusan masalah dari tugas akhir
Rumusan masalah yang diangkat dalam penelitian ini adalah sebagai berikut: 
\begin{enumerate}
  \item Bagaimana implementasi model StrongSORT untuk meningkatkan akurasi dalam menghitung jumlah kendaraan yang melintas?
  \item Bagaimana meningkatkan akurasi model StrongSORT dalam menghitung jumlah kendaraan yang melintas?
\end{enumerate}

\section{Batasan Masalah}

% Ubah paragraf berikut sesuai dengan penelitian lain yang terkait dengan tugas akhir
Batasan masalah yang digunakan dalam penelitian ini adalah sebagai berikut: \begin{enumerate}
  \item Pengambilan data dilakukan pada pagi hari, siang hari dan sore hari
  \item Data pengujian yang digunakan sebagai bahan pengolahan citra berupa rekaman video CCTV pada kota Surabaya
  \item Kendaraan yang dapat dihitung terbatas pada 5 kategori kendaraan, yaitu Sepeda, Motor, Mobil, Bus dan truk
  \item Peningkatan akurasi akan dibandingkan dengan sistem penghitungan kendaraan dengan YOLOv7 tanpa menggunakan StrongSORT sebagai MOT
\end{enumerate}

\section{Tujuan Penelitian}

% Ubah paragraf berikut sesuai dengan tujuan penelitian dari tugas akhir
Tujuan penelitian yang akan dicapai dalam penelitian ini adalah sebagai berikut:
\begin{enumerate}
    \item Mengimplementasikan model StrongSORT untuk meningkatkan akurasi penghitungan jumlah kendaraan terhadap kendaraan yang melintas.
    \item Mengetahui hasil berupa peningkatan akurasi dari penerapan model StrongSORT dalam menghitung jumlah kendaraan yang melintas.
\end{enumerate}

\section{Manfaat Penelitian}
Terdapat dua manfaat yang dihasilkan dalam penelitian ini yaitu secara teori dan pratik. Secara teori, penelitian ini memberikan kontribusi penelitian pada bidang \emph{computer vision} yang dapat digunakan sebagai referensi pada penelitian berikutnya. Secara praktik, penelitian ini dapat berkontribusi pada \emph{Intelligent Transportation System} dalam mengumpulkan data berupa jumlah kendaraan yang melintas pada suatu jalan.
\newpage 


  % Konten tinjauan pustaka
  \chapter{TINJAUAN PUSTAKA}


\section{Penelitian Terdahulu}
\lipsum[2-4]


\subsection{ResNeSt50}
\lipsum[2-4]

\begin{figure}[H]
  \centering
  \includegraphics[width=0.5\textwidth]{./gambar/resnest-arsitektur.png}
  \caption{Arsitektur RestNeSt50}
  \label{fig:IoUequation}
\end{figure}



\newpage


  % Konten metodologi
 \chapter{METODOLOGI}

% Ubah konten-konten berikut sesuai dengan isi dari metodologi
Pada bab ini akan dijelaskan langkah-langkah dalam mengerjakan penelitian ini. Metodologi penelitian digunakan sebagai acuan penelitian agar dapat berjalan secara sistematik. Langkah-langkah yang akan dilaksanakan pada penelitian ini ditunjukkan pada blok diagram.

\begin{figure}[ht]
	\centering
	\includegraphics[scale=0.85]{./gambar/diagramrev.pdf}
	\caption{Blok Diagram Penelitian}
	\label{fig:diagrammetodologi}
\end{figure}

\section{Pengumpulan Data}
Data yang digunakan pada penelitian untuk model deteksi objek merupakan kumpulan data berupa gambar. Dataset yang digunakan diambil dari dataset publik seperti COCO \autocite{DBLP:journals/corr/LinMBHPRDZ14} yang terdiri dari 47.202 gambar, OpenImage \autocite{DBLP:journals/corr/abs-1811-00982} yang terdiri dari 147.699 gambar, dan video CCTV di kota Surabaya dengan resolusi SD(640x480) dan HD (1280x720), dimana pada video tersebut akan dilakukan ekstraksi setiap 10 framenya.

\section{Analisis dan Perancangan Sistem}
Pada tahapan ini, dilakukan percangan sistem dimulai dari penggunaan bahasa pemrograman hingga tahap perhitungan jumlah kendaraan yang melintas. Sistem ini menggunakan bahasa pemrograman Python dan \emph{library} PyTorch.
\begin{enumerate}
\item \emph{Pre-processing Data}

Gambar yang diperoleh dari video CCTV akan dilakukan \emph{image cropping} yang bertujuan agar memfokuskan gambar pada objek kendaraan saja. Langkah selanjutnya dilakukan tahapan pemberian label. Label akan dibagi kedalam 5 kategori kendaraan, yaitu Sepeda, Motor, Mobil, Bus dan truk. Langkah terakhir dalam \emph{pre-processing data} adalah pembagian data latih dan data uji. Data set dibagi menjadi 2 bagian dengan komposisi 80\% dari seluruh gambar untuk melatih model, dan sisa 20\% untuk melakukan uji coba terhadap hasil dari model.

\item Pelatihan Model Deteksi Objek

Hasil data latih dari tahapan \emph{pre-processing data} akan digunakan sebagai input dalam melatih model pendeteksi objek berbasis YOLOv7. Dalam pelatihan ini nantinya bobot awal yang digunakan akan menggunakan data \emph{pre-trained weight} yang telah dilatih sebelumnya pada jaringan YOLO untuk suatu dataset dalam hal ini akan digunakan \emph{pre-trained model} Common Objects in Context (COCO). \emph{Pre-trained weight} menjadikan proses dari training \emph{model weight} memakan waktu yang lebih singkat. Hasil dari proses ini adalah sebuah bobot baru dari hasil data training sebelumnya atau disebut \emph{Trained weight}. \emph{Trained weight} inilah nantinya yang akan digunakan dalam pendeteksian objek kendaraan.

\item Pelacakan Objek Kendaraan

Pada tahap ini akan mengiplementasikan metode MOT dengan algoritma StrongSORT. Hasil dari pelatihan model deteksi objek pada YOLOv7 kemudian akan digunakan untuk mendeteksi objek kendaraan dan mengklasifikasi jenisnya. Secara berurutan, setiap \emph{frame} dari video akan masuk ke dalam jaringan YOLOv7 untuk dilakukan pendeteksian objek, kemudian didapatkan keluaran berupa himpunan koordinat (\emph{bbox}), kelas objek (\emph{cls}) dan nilai keyakinan dari setiap objek yang muncul pada \emph{frame} (\emph{conf}), nilai-nilai ini kemudian akan diproses dengan algoritma StrongSORT. Kemudian StrongSORT akan memberikan id terhadap objek kendaraan yang terdeteksi oleh YOLOv7 dan terus melacaknya selama objek tersebut masih berada dalam \emph{frame} video. StrongSORT melacak objek dengan memprediksi benda menggunakan NSA Kalman dan mencocokan kemiripan objek berdasarkan \emph{appearance feature extractor} dari model BoT + ResNest yang sudah dilatih sebelumnya.

\item Implementasi StrongSORT
 \begin{enumerate}
\item{ResNeSt50}\\
ResNeSt50 akan digunakan sebagai backbone pada \emph{appearance feature extractor} yang dilatih dengan dataset VeRi. Dimana ResNest50 akan dapat melakukan \emph{appearance feature extractor} untuk mencocokan objek atau melakukan reidentifikasi. 


\begin{figure}[H]
  \centering
  \includegraphics[width=0.5\textwidth]{./gambar/resnest-arsitektur.png}
  \caption{Arsitektur RestNeSt50}
  \label{fig:IoUequation}
\end{figure}

\item{Exponential Moving Average}\\
Dalam \emph{feature updating} akan dilakukan mengikuti EMA dimana akan memperbaharui \emph{appearance state} $a_i^t$ untuk $i-th$ \emph{tracklet} pada frame $t$.
\begin{equation}
e_i^t={\alpha e_i^{t-1} +  \left(1-\alpha\right) f_i^t}
\end{equation}
Dimana:\\
\hspace*{3em}
\begin{tabular}{rl}
    $e_i^t$:& pembaharuan \emph{appearance embedding} untuk $i-th$ \emph{tracklet} pada frame $t$ \\
    $ \alpha$:& \emph{momentum term} dimana $ \alpha = 0.9$  \\
    $  f_i^t$:&\emph{appearance embedding} dari deteksi yang cocok saat ini. \\
\end{tabular}
\\

\item{Enhanced Correlation Coefficient (ECC)}

ECC digunakan sebagai \emph{camera motion compensation} untuk menstabilkan gambar akibat dari pergerakan kamera.

\item{\emph{Noise Scale Adaptive} (NSA) Kalman Filter}\\
NSA Kalman Filter akan digunakan untuk memprediksi posisi berikutnya berdasarkan dari posisi dan kecepatan benda. Pada NSA Kalman Filter \emph{noise covariance} dihitung dengan persamaan berikut:
\begin{equation}
\tilde{R}_k=\left(1-c_k\right) R_k
\end{equation}
Dimana:\\
\hspace*{3em}
\begin{tabular}{rl}
    ${R}_k$:& \emph{constant measurement noise covariance}. \\
    $ c_k$:& \emph{confidence score} pada keadaan $k$. \\
\end{tabular}
\\
\begin{figure}[H]
  \centering
  \includegraphics[width=0.8\textwidth]{./gambar/kalmanposition.png}
  \caption{NSA Kalman Filter memprediksi pergerakan benda}
  \label{fig:IoUequation}
\end{figure}

\end{enumerate}
\item Penghitungan Kendaraan

Pemberian id oleh StrongSORT pada objek kendaraan yang melintas akan dimanfaatkan untuk menghitung jumlah kendaraan. Jika id dari objek adalah id yang baru maka akan dihitung sebagai kendaraan baru dan ditambahkan pada penghitung kendaraan sesuai dengan kategorinya berdasarkan klasifikasi dari pendeteksi objek YOLOv7.

\end{enumerate}
\section{Simulasi Sistem}
Pada tahap ini dilakukan implementasi dari tugas akhir berupa simulasi sistem sesuai dengan perancangan sistem yang dapat menghitung jumlah kendaraan yang melintasi jalan tertentu yang nantinya akan dijalankan dengan data uji berupa video CCTV lalu lintas yang jumlah kendaraannya sudah dihitung secara manual.
\newpage
\section{Desain Uji Coba Sistem}
\begin{figure}[H]
	\centering
	\includegraphics[scale=0.85]{./gambar/diagramujicobasistem.pdf}
	\caption{Blok Diagram Uji Coba Sistem}
	\label{fig:diagramujicobasistem}
\end{figure}

Setelah melakukan simulasi sistem, pada tahap ini akan membandingkan hasil akurasi dari model penghitungan menggunakan YOLOv7 dan StrongSORT yang diusulkan pada penelitian ini dengan model penghitungan yang hanya menggunakan YOLOv7. Untuk melakukan hal tersebut, maka pada penelitian ini juga dikembangkan model penghitungan kendaraan hanya dengan menggunakan YOLOv7. Kedua model menjalankan uji coba dengan menggunakan \emph{dataset} yang sama. Gambar \ref{fig:diagramujicobasistem} menunjukkan skema desain uji coba sistem.

\section{Analisis dan Evaluasi Hasil}
Tahap selanjutnya dari penelitian Tugas Akhir ini adalah 
menganalisis hasil simulasi program yang telah dibuat untuk 
menguji apakah program yang dibuat dapat menghitung jumlah kendaraan yang melintas dengan baik. Pengujian dilakukan dengan melihat perhitungan kendaraan menggunakan StrongSORT dan YOLOv7. Selain itu akan dicari seberapa akurat metode tersebut dapat menghitung kendaraan yang melintas dan juga dilakukan uji empiris yaitu dengan cara membandingkan perhitungan antara sistem dan hitung manual. Apabila terdapat beberapa fungsi dan hasil yang belum sebagaimana mestinya maka dari tahap analisis ini diharapkan dapat diketahui kekurangannya sehingga dapat diperbaiki.


\section{Penarikan Kesimpulan dan Saran}

Pada tahap terakhir, akan dilakukan penarikan kesimpulan dan saran dari penelitian ini. Setelah itu akan dilakukan pembukuan tugas akhir sebagai hasil dari laporan penelitian yang telah dikerjakan.
\newpage

  % Konten lainnya
  \section{Jadwal Kegiatan}
Berikut ini disajikan tabel jadwal kegiatan yang akan dilakukan selama 3 bulan yang berkoresponden dengan metode penelitian.
% Ubah tabel berikut sesuai dengan isi dari rencana kerja
\newcommand{\w}{}
\newcommand{\G}{\cellcolor{gray}}
\begin{table}[h!]
  \begin{tabular}{|p{3.5cm}|c|c|c|c|c|c|c|c|c|c|c|c|c|c|c|c|}

    \hline
    \multirow{2}{*}{Nama Kegiatan} & \multicolumn{16}{|c|}{Minggu Ke-} \\
    \cline{2-17} &
    1 & 2 & 3 & 4 & 5 & 6 & 7 & 8 & 9 & 10 & 11 & 12 & 13 & 14 & 15 & 16 \\
    \hline

    % Gunakan \G untuk mengisi sel dan \w untuk mengosongkan sel
    Pengumpulan Data &
    \G & \G & \w & \w & \w & \w & \w & \w & \w & \w & \w & \w & \w & \w & \w & \w \\
    \hline

    Analisis dan Perancangan Sistem &
    \w & \w & \G & \G & \G & \G & \G & \G & \G & \w & \w & \w & \w & \w & \w & \w \\
    \hline
    
    Simulasi Sistem &
    \w & \w & \w & \w & \w & \w & \w & \w & \w & \G & \w & \w & \w & \w & \w & \w \\
    \hline
    
    Desain Uji Coba Sistem &
    \w & \w & \w & \w & \w & \w & \w & \w & \w & \w & \G & \w & \w & \w & \w & \w \\
    \hline

    Analisis dan Evaluasi Hasil&
    \w & \w & \w & \w & \w & \w & \w & \w & \w & \w & \w & \G & \G & \G & \w & \w \\
    \hline
    
    Penarikan Kesimpulan dan Saran &
    \w & \w & \w & \w & \w & \w & \w & \w & \w & \w & \w & \w & \w & \w & \G & \G \\
    \hline

  \end{tabular}
\end{table}

\newpage


  
  %\input{konten/5-dapus}
  

  % Daftar pustaka
  \chapter{DAFTAR PUSTAKA}
 
  % \renewcommand\refname{}
  % \vspace{-2ex}
  \printbibliography[heading=none]
  %\bibliography{./pustaka/pustaka.bib}

\end{document}
