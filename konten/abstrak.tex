\begin{center}
  \large\textbf{Peningkatan Akurasi Perhitungan Jumlah Kendaraan menggunakan model StrongSORT}
\end{center}

\addcontentsline{toc}{chapter}{ABSTRAK}

\vspace{2ex}


\begingroup
  \setlength{\tabcolsep}{0pt}
  \noindent
  \textbf{
  \begin{tabularx}{\textwidth}{l >{\centering}m{2em} X}
    Nama Mahasiswa / NRP        &:& I Gusti Agung Vivekananda / 06111940000111 \\
    Departemen       &:&	Matematika FSAD - ITS \\
    Dosen Pembimbing  &:& Dr. Budi Setiyono, S.Si, MT \\
  \end{tabularx}
  }
\endgroup

\vspace{6ex}

\noindent\textbf{Abstrak}
\vspace{1ex}


% Ubah paragraf berikut dengan abstrak dari tugas akhir
Perkembangan jumlah kendaraan di Indonesia terus meningkat. Hal ini menyebabkan kemacetan dan penambahan jumlah emisi yang menjadi kontributor utama terhadap perubahan iklim global. Salah satu cara menyelesaikan permasalahan tersebut adalah dengan menerapkan \emph{Intelligent Transportation Systems} (ITS). Perhitungan jumlah kendaraan merupakan komponen penting dalam ITS karena jumlah kendaraan memberikan informasi penting mengenai arus lalu lintas seperti kondisi lalu lintas, okupansi lajur, dan tingkat kemacetan, yang dapat dimanfaatkan untuk peringatan kecelakaan, dan pencegahan kemacetan. Penelitian ini mengusulkan untuk mengembangkan penghitungan jumlah kendaraan dengan menggunakan model StrongSORT untuk melakukan pelacakan pada objek kendaraan dan YOLOv7 untuk mendeteksi objek kendaraan. Data latih yang digunakan untuk pendeteksi objek diperoleh dari \emph{Common Objects in Context} (COCO), OpenImage, dan CCTV di kota Surabaya. Hasil penelitian ini akan dibandingkan dengan model penghitungan kendaraan menggunakan YOLOv7.

\vspace{4ex}

% Ubah kata-kata berikut dengan kata kunci dari tugas akhir
\noindent\textbf{Kata Kunci: YOLOv7, StrongSORT, \emph{Computer Vision}, \emph{Multi Object Tracking}, \emph{Intelligent Transportation Systems}}
